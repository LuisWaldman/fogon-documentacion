\subsubsection{Vista de procesos}\label{sec:vista-procesos}

Aqui repasaremos, con diagramas de secuencia, como la clase reproductor utiliza
un Strategy para manejar la reproduccion en modo desconectado y en modo conectado. 

Ejemplificamos con el proceso IniciarReproduccion, pero debe suponerse 
uno similar para DetenerReproduccion,CambiarCompas, CambiarLaListaDeCanciones, etc.

Luego, mostraremos la secuencia con la que logra sincronizar relojes con el servidor y
entre clientes usando WebRTC.

\begin{figure}[H]
\centering
\includegraphics[width=0.8\textwidth]{out/diaguml/diagrama-clase-cancion/diagrama-secuencia-iniciarcancion}
\caption{Diagrama de de secuencia de sincronizacion}
\end{figure}



\begin{figure}[H]
\centering
\includegraphics[width=0.8\textwidth]{out/diaguml/diagrama-clase-cancion/diagrama-secuencia-iniciarcanciondistribuido.}
\caption{Diagrama de de secuencia de sincronizacion}
\end{figure}


\paragraph{IniciarReproduccion}


\begin{figure}[H]
\centering
\includegraphics[width=0.8\textwidth]{out/diaguml/diagrama-clase-cancion/diagrama-secuencia-sincronizar}
\caption{Diagrama de de secuencia de sincronizacion}
\end{figure}

