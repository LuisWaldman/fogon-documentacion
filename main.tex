\documentclass[12pt,a4paper]{article}

% Paquetes necesarios
\usepackage[utf8]{inputenc}
\usepackage[spanish]{babel}
\usepackage{graphicx}
\usepackage{amsmath}
\usepackage{amssymb}
\usepackage{hyperref}
\usepackage{geometry}
\usepackage{fancyhdr}
\usepackage{setspace}
\usepackage{caption}
\usepackage{listings}
\usepackage{xcolor}

% Configuración de la página
\geometry{
    left=3cm,
    right=2.5cm,
    top=2.5cm,
    bottom=2.5cm
}

% Configuración de encabezados y pie de página
\pagestyle{fancy}
\fancyhf{}
\rhead{\thepage}
\lhead{Trabajo Práctico Final}

% Configuración de código fuente
\lstset{
    basicstyle=\ttfamily\small,
    breaklines=true,
    frame=single,
    numbers=left,
    numberstyle=\tiny\color{gray},
    keywordstyle=\color{blue},
    commentstyle=\color{green!60!black},
    stringstyle=\color{red}
}

% Información del documento
\title{Trabajo Práctico Profesional\\
\large{El fogon}}
\author{Luis Waldman\\
\small{Legajo: 79279}\\
\small{Ingeniería en Informática}}
\date{\today}

\begin{document}

% Portada
\maketitle
\thispagestyle{empty}

\begin{center}
\vspace{2cm}
% \includegraphics[width=0.3\textwidth]{logo.png} % Descomentar cuando tengas el logo
\vspace{2cm}

\textbf{Universidad de Buenos Aires}\\
\vspace{0.5cm}
Año 2025
\end{center}

\newpage

% Índice
\tableofcontents
\newpage

% Resumen
\section*{Resumen}
\addcontentsline{toc}{section}{Resumen}
Breve descripción del trabajo realizado, objetivos principales y conclusiones más importantes.

\newpage

% Introducción
\section{Introducción}
\label{sec:introduccion}

En esta sección se presenta el contexto del trabajo, la problemática a resolver y los objetivos que se persiguen.

\subsection{Motivación}
Describir qué motivó la realización de este trabajo.

\subsection{Objetivos}
\subsubsection{Objetivo General}
Describir el objetivo general del trabajo.

\subsubsection{Objetivos Específicos}
\begin{itemize}
    \item Objetivo específico 1
    \item Objetivo específico 2
    \item Objetivo específico 3
\end{itemize}

% Marco Teórico
\section{Marco Teórico}
\label{sec:marco-teorico}

Desarrollar los conceptos teóricos necesarios para comprender el trabajo.

\subsection{Conceptos Fundamentales}
Explicar los conceptos básicos relacionados con el tema.

\subsection{Estado del Arte}
Describir trabajos previos relacionados y el estado actual del conocimiento en el área.

% Metodología
\section{Metodología}
\label{sec:metodologia}

Describir la metodología empleada para el desarrollo del trabajo.

\subsection{Herramientas Utilizadas}
\begin{itemize}
    \item Herramienta 1
    \item Herramienta 2
    \item Herramienta 3
\end{itemize}

\subsection{Procedimiento}
Detallar los pasos seguidos para llevar a cabo el trabajo.

% Desarrollo
\section{Desarrollo}
\label{sec:desarrollo}

Presentar el desarrollo del trabajo de manera detallada.

\subsection{Implementación}
Describir cómo se implementó la solución propuesta.

% Ejemplo de inserción de código
\begin{lstlisting}[language=Python, caption=Ejemplo de código]
def funcion_ejemplo():
    print("Hola, mundo!")
    return True
\end{lstlisting}

% Ejemplo de figura
\begin{figure}[h]
    \centering
    % \includegraphics[width=0.8\textwidth]{imagen.png}
    \caption{Descripción de la figura}
    \label{fig:ejemplo}
\end{figure}

% Ejemplo de tabla
\begin{table}[h]
    \centering
    \begin{tabular}{|c|c|c|}
        \hline
        \textbf{Columna 1} & \textbf{Columna 2} & \textbf{Columna 3} \\
        \hline
        Dato 1 & Dato 2 & Dato 3 \\
        \hline
        Dato 4 & Dato 5 & Dato 6 \\
        \hline
    \end{tabular}
    \caption{Descripción de la tabla}
    \label{tab:ejemplo}
\end{table}

% Resultados
\section{Resultados}
\label{sec:resultados}

Presentar los resultados obtenidos del trabajo realizado.

\subsection{Análisis de Resultados}
Analizar e interpretar los resultados obtenidos.

% Ejemplo de ecuación
\begin{equation}
    E = mc^2
    \label{eq:einstein}
\end{equation}

% Conclusiones
\section{Conclusiones}
\label{sec:conclusiones}

Presentar las conclusiones del trabajo, resumiendo los principales hallazgos y logros alcanzados.

\subsection{Trabajo Futuro}
Describir posibles extensiones o mejoras que podrían realizarse en el futuro.

% Referencias
\begin{thebibliography}{99}
\bibitem{referencia1}
Autor, A. (2024). \textit{Título del libro o artículo}. Editorial.

\bibitem{referencia2}
Autor, B. y Autor, C. (2023). Título del artículo. \textit{Nombre de la Revista}, 10(2), 123-145.

\bibitem{referencia3}
Autor, D. (2025). Documento en línea. Disponible en: \url{https://ejemplo.com}
\end{thebibliography}

% Apéndices (opcional)
\appendix
\section{Apéndice A: Material Adicional}
\label{app:material-adicional}

Incluir material adicional que complemente el trabajo pero que no es esencial para su comprensión.

\end{document}
