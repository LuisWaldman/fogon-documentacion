\subsubsection{Tocar}

La página tocar muestra la vista según el músico:

\begin{figure}[H]
\centering
\includegraphics[width=0.8\textwidth]{imagenes/TocandoMuchos}
\caption{Página tocar con distintos instrumentos en varios exploradores}
\end{figure}

En la cabecera, muestra los datos de la canción y permite cambiar la vista y la configuración.

A lo largo de la pantalla, la vista particular del instrumento.

Debajo, el control de reproducción y el metrónomo.

\paragraph{icono-fogon}

Arriba a la izquierda, el icono del fogón muestra el estado general de la aplicación. Marca el pulso cuando 
están tocando y muestra los usuarios en la sesión.

\begin{figure}[H]
\centering
\includegraphics[width=0.5\textwidth]{imagenes/iconofogones}
\caption{Iconos para los estados: desconectado, conectado, logueado, o en sesión.}
\end{figure}

\paragraph{Menú}

Arriba a la derecha, el icono de menú abre las opciones para configurar la vista y el usuario.

\paragraph{Vistas}


Para configurar su vista, cada músico accede desde el menú a ``Ver'':

\begin{figure}[H]
\centering
\includegraphics[width=0.6\textwidth]{imagenes/configuracionvista}
\caption{Configuración de la vista}
\end{figure}

Desde aquí puede elegir entre ver letras y/o acordes o la partitura. 
Además, puede ajustar el tamaño de la letra, los acordes y la partitura para ajustarla a cada dispositivo.

En la opción de reproducción, puede elegir acompañar la reproducción con un video o con un MIDI generado a través de 
las partituras.

En las últimas opciones, se ajusta la cantidad de columnas que se muestra en pantalla y 
si muestran instrucciones para el músico, la secuencia de acordes o la pantalla para reproducir MIDIs.

