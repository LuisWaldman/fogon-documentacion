\subsubsection{tocar}

La pagina tocar muestra la vista segun el musico:

[ACA, FOTO DE VARIAOS EXPLORADORES MOSTRANODO 
LA PAGINA TOACAR CON DISTINTOS INSTRUMENTOS]

En la cabecera, muestra los datos de la cancion y permite cambiar la vista y la Configuración.

A lo largo de la pantalla, la vista particular del instrumento.

Debajo, el control de reproduccion y el metrónomo.

\paragraph{icono-fogon}

Arriba a la izquierda, el icono del fogon muestra el estado general de la aplicación. Marca el pulso cuando 
estan tocando y muestra los usuarios en la sesión.

[ACA, FOTO DEL ICONO FOGON]

\paragraph{menu}

Arriba a la derecha, el icono de menu abre las opciones para configurar la vista y el usuario.

\paragraph{vistas}


Para configurar su vista, cada musico accede desde el menu a "Ver":

[ACA, FOTO DEL MENU DESPLEGADO VER]

Desde aqui puede elegir entre ver letras y/o acordes o la partitura. 
Ademas, puede ajustar el tamaño de la letra, los acordes y la partitura para ajustarla a cada dispositivo.

En la opcion de reproduccion, puede elegir acompañar la reproduccion con un video o con un midi generado a travez de 
las partituras.

En las ultimas opciones, se ajuste la cantidad de columnas que se muestra en pantalla y 
si muestran instrucciones para el musico, la secuencia de acordes o la pantalla para reproducir MIDIs.

