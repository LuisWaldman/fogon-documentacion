\subsubsection{Vista Física}
\label{sec:vista-fisica}

La vista física describe cómo los componentes de software se mapean en la infraestructura de hardware y la topología de red del sistema.

\paragraph{Arquitectura de Despliegue}

El sistema El Fogón utiliza una arquitectura cliente-servidor con capacidad de operación offline:

\begin{itemize}
    \item \textbf{Nodos Cliente:} Dispositivos heterogéneos (smartphones, tablets, laptops, desktops) que ejecutan la PWA en navegadores web modernos.
    
    \item \textbf{Servidor Central:} Servidor Golang que hospeda:
    \begin{itemize}
        \item Servidor HTTP/HTTPS para servir la aplicación web
        \item Servidor WebSocket para sincronización en tiempo real
        \item API REST para operaciones CRUD
        \item Base de datos de canciones y usuarios
    \end{itemize}
    
    \item \textbf{CDN (Opcional):} Para distribución de assets estáticos (imágenes, CSS, JS).
\end{itemize}

\paragraph{Topología de Red}

\textbf{Topología Estrella:} Todos los clientes se conectan al servidor central. El servidor actúa como hub de comunicación para las sesiones sincronizadas.

\textbf{Ventajas:}
\begin{itemize}
    \item Simplicidad en la gestión de estado compartido
    \item El servidor es la única fuente de verdad
    \item Facilita la resolución de conflictos
\end{itemize}

\textbf{Desventajas:}
\begin{itemize}
    \item Punto único de falla
    \item Latencia agregada (cliente A → servidor → cliente B)
\end{itemize}

\textbf{Mitigación con WebRTC:} En desarrollo - topología híbrida donde los clientes pueden conectarse peer-to-peer para reducir latencia en LANs, manteniendo el servidor como fallback.

\paragraph{Distribución de Responsabilidades}

\begin{itemize}
    \item \textbf{En el Cliente:}
    \begin{itemize}
        \item Renderizado de vistas (Vue.js + VexFlow)
        \item Lógica de presentación y UI
        \item Cache local de canciones (IndexedDB)
        \item Service Worker para operación offline
        \item Compensación de jitter y scheduling preciso de eventos
    \end{itemize}
    
    \item \textbf{En el Servidor:}
    \begin{itemize}
        \item Autoridad sobre el estado compartido
        \item Persistencia de canciones y usuarios (base de datos)
        \item Broadcast de eventos de sincronización
        \item Autenticación y autorización
        \item Mediación de timestamps para sincronización
    \end{itemize}
\end{itemize}

\paragraph{Requisitos de Hardware y Red}

\begin{itemize}
    \item \textbf{Cliente:} Navegador moderno (Chrome 90+, Firefox 88+, Safari 14+), 2GB RAM mínimo, conexión a internet opcional.
    
    \item \textbf{Servidor:} CPU de 2+ cores, 4GB RAM, 20GB almacenamiento, conexión de banda ancha (10 Mbps mínimo).
    
    \item \textbf{Red:} Para sincronización en tiempo real, se recomienda latencia < 100ms y conexión estable. El sistema adapta el buffer según latencia detectada.
\end{itemize}

\paragraph{Escalabilidad}

El servidor Golang puede escalar horizontalmente con:
\begin{itemize}
    \item Balanceador de carga para HTTP/HTTPS
    \item Sticky sessions para WebSocket
    \item Redis para compartir estado entre instancias del servidor
\end{itemize}

% TODO: Agregar diagrama de despliegue UML
% \begin{figure}[h]
%     \centering
%     \includegraphics[width=\textwidth]{imagenes/diagrama-despliegue.png}
%     \caption{Diagrama de despliegue del sistema El Fogón}
%     \label{fig:despliegue}
% \end{figure}
cómo se despliega en hardware (nodos, servidores, redes).
\begin{itemize}
    \item Componente 1
    \item Componente 2
    \item Componente 3
\end{itemize}
