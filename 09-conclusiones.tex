\section{Concluciones}
\label{sec:concluciones}

Resumen final del trabajo, destacando los objetivos alcanzados y el valor del sistema desarrollado.

\subsection{Desarrollos Futuros}

El sistema esta pensado para la escalabilidad, en modo "desarrollador", 
en las opciones de Configuración se puede definir el servidor al que se conecta.
Si el fogon tiene exito mundial, nos bastaria con agregar mas servidores 
distribuidos por region e implementar un paso previo en el que el cliente
se conecta con un balanceador de carga que lo redirige al servidor mas cercano.

El diseño de la aplicacion, y principalmente Vue..js, permite agregar 
nuevas vistas sincronizadas. Podriamos agregar alguna que muestre
el teclado con las notas bajando y alguna vista para bateristas.

Otra funcionalidad que quedó pendiente, es la de que la aplicacion no avanze 
el tiempo según transcurre, si no según escucha lo que escucha que el musico esta tocando.
Asi, podria ayudar "esperando" a un aprendiz.

Nos queda agregar algún modo de editar ritmo como acompañamiento, que sumado 
a los pentagramas de VexFlow, permitiria editar canciones completas.

Aun hace falta sumar mas repertorio. Durante el desarrollo se crearon 
varias canciones y algunos algoritmos para obtener los acordes desde 
un cancionero, y las letras de las canciones desde Youtube. 

Igual, hace falta crear mas canciones con letra + acordes + partitura + video 
que permitan el uso completo de la aplicacion.

\subsection{Lecciones aprendidas}

TDD desde el principio

La IA es barbara, loco.


\subsection{Conclusiones}

El proyecto fue ambicioso, y se logró desarrollar un sistema funcional que cumple con los objetivos planteados.
