\section{Concluciones}
\label{sec:concluciones}

Resumen final del trabajo, destacando los objetivos alcanzados y el valor del sistema desarrollado.

\subsection{Desarrollos Futuros}

El sistema esta pensado para la escalabilidad, en modo "desarrollador", 
en las opciones de Configuración se puede definir el servidor al que se conecta.
Si el fogon tiene exito mundial, nos bastaria con agregar mas servidores 
distribuidos por region e implementar un paso previo en el que el cliente
se conecta con un balanceador de carga que lo redirige al servidor mas cercano.

El diseño de la aplicacion, y principalmente Vue..js, permite agregar 
nuevas vistas sincronizadas. Podriamos agregar alguna que muestre
el teclado con las notas bajando y alguna vista para bateristas.

Otra funcionalidad que quedó pendiente, es la de que la aplicacion no avanze 
el tiempo según transcurre, si no según escucha lo que escucha que el musico esta tocando.
Asi, podria ayudar "esperando" a un aprendiz.

Nos queda agregar algún modo de editar ritmo como acompañamiento, que sumado 
a los pentagramas de VexFlow, permitiria editar canciones completas.

Aun hace falta sumar mas repertorio. Durante el desarrollo se crearon 
varias canciones y algunos algoritmos para obtener los acordes desde 
un cancionero, y las letras de las canciones desde Youtube. 

Igual, hace falta crear mas canciones con letra + acordes + partitura + video 
que permitan el uso completo de la aplicacion.

\subsection{Lecciones aprendidas}

Durante el desarrollo del proyecto, el año 2025, el uso de la IA en programación 
paso de ser recomendada a inevitable. A fines de septiembre, cuando empezaba a programar una funcion,
ya aparecia sugerida, a veces mejor de lo que pensé, a veces antes de que lo piense.

Como todos, caí en la tentacion de hacer 'no-code' en algunas funcionalidad: traspasar tonalidad de canciones, 
contar versos en poemas, pasar el audio a notas, etc. Pero cuando algún problema empezaba a ser complejo,
una acción de la IA deshacia lo que habia hecho anteriormente. Lo mismo pasaba con los humanos 
y ya teniamos una solución: guiar el desarrollo con pruebas.

Durante la construccion del proyecto el modelo claude sonet 4.5, guiado por pruebas (que hasta a veces escribia con el mismo),
intervino en el desarrollo de la mayoria de las funcionalidades y en la corrección de bugs.

Sobre el final del proyecto, cuando este tenia distintas vistas y estados, se hizo necesario un modo de probar la aplicación en
 su conjunto. Sobre el final del desarrolló, armé test punta a punta con Playwright; pero si tendria que rehacer el 
 proyecto en 2026, lo haria desde el principio con pruebas de integración, o sea: desarrollo guiado por comportamiento.

  La IA va a seguir tomando lugar: en este momento que escribo, si dejo 
 de tipear un ratito, una IA me sugiere como terminar la frase. 
 El profesor que lea algun trabajo profesional 
 seguro lo va a pasar por un filtro con IA y el alumno que lo presenta tambien. 
 Esto nos deja a todos un elevado nuevo piso y nos posibilita alturas mayores.

 Aves de mal aguero anuncian un cambio en el estado del arte cada 1 minuto, 
 porque asumen al humano totalmente ausente de las desiciones tecnologicas.
 Pero si el humano quiere seguir guiando el desarrollo, lo va a tener que hacer a travez de una arquitectura clara y 
 de pruebas que validen el comportamiento.

 \subsection{Conclusiones}

El proyecto fue ambicioso, no solo queria que el proyecto sea novedoso, 
sino que el conjunto de tecnologias utilizadas sea el ideal y que 
reflejara lo que aprendí en la facultad.

Con respecto a las herramientas, el framework vitejs, con el codigo regulado por 'Lint' y pruebas unitarias en una integracion continua que termina 
desplegandose en el fogon.ar; terminó resultando ser una excelente eleccion. Sobre el servidor, 
Golang hizo facil crear APIs y procesos multihilos para atender a los clientes; y tanto en  Render.com como Fly.io, 
pudimos hacer despliegues automaticos.

Sobre lo aprendido en la facultad, En Tecnicas de Diseño, dejamos de " solo programar " para empezar a pensar los sistemas por su arquitectura 
y a ordenar los problemas por los patrones que lo resuelven. Luego, en Introducción a Sistemas Distribuidos 
y en Sistemas Distribuidos entendimos que es un protococolo, como calibrar relojes, que es y como se elige un proceso lider. 

Esto lo utilicé para pensar los desafios principales en la aplicación. Para el pentagrama en VexFlow, por ejemplo,
decidimos usar el patron Facade en vez de Proxy, y al revez para administrar el video; las multiples vistas son rehutilizables y extensibles
 porque estan separadas del modelo y del controlador; El estado compartido del Fogon, es pensado como "estado compartido" porque toda la aplicacion esta pensada en los terminos 
aprendidos en Sistemas Distribuidos.

Ademas, desarrollé ademas otras herramientas para medir y validar la exactitud de la calibracion, 
lo que muestra un rigor cientifico caracteristico de la academia.

Por todo esto creo que el fogon.ar suma a los conocimientos adquiridos en FIUBA una mirada original,
, llevada a cabo con herramientas del estado del arte actual, con rigor academico y cientifico, 
y con un resultado que sobrepasa los objetivos.