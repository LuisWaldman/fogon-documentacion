\section{Estado del Arte}
\label{sec:estado-arte}

Como es parte nuestro "negocio", describiremos algunos conceptos sobre la naturaleza
 del sonido y de la música. 
 
 Luego, comentaremos algunas tecnologías destacadas que 
 tomamos en el Fogón y sobre las novedades que llegaron con internet. También,
 repasaremos aplicaciones web con IA que ofrecen servicios relacionados con la música.

  Finalmente, explicaremos cómo nuestro enfoque es novedoso y 
  aporta valor, comparado con las herramientas antes mencionadas.

\subsection{Sobre el sonido y la música}

"El sonido es una onda mecánica que se propaga a través de un medio elástico, como el aire o el agua."
y que sus características principales son la frecuencia, la amplitud y el timbre.

La frecuencia determina el tono o la nota musical, 
la amplitud el volumen y el timbre la calidad del sonido, permite distinguir entre diferentes instrumentos que tocan la misma nota.

La mayoría de los humanos no podemos determinar 
la frecuencia a la que vibra una cuerda, pero si podemos distinguir si produce un sonido agradable.
Pitágoras descubrió, en el siglo VI a.C., que para producir sonidos agradables entre dos cuerdas vibrantes 
tienen que tener relaciones matemáticas. Ej: (2:1)Una octava, 
 (3:2)una quinta justa. (4:3)una cuarta justa.

Todos conocemos la definición de música acerca de hacer algo con la melodía, la armonía y el ritmo.
Pues bien, la melodía es cómo cambia la frecuencia en el tiempo, la armonía es la combinación de varias frecuencias.
El ritmo es la repetición de sonidos (o volúmenes o intenciones) en intervalos regulares. 
En la música suelen organizarse de a 
2, 3 o 4 tiempos, formando compases.


\subsection{Herramientas}

\subsubsection{Pentagramas}


decíamos que Pitágoras descubrió que para
producir sonidos agradables entre dos cuerdas vibrantes 
tienen que tener relaciones matemáticas. Pero ya 15 siglos, por el 2000 A.C.,
 en la Mesopotamia, se usaban tablillas de arcilla para anotar música, 
con símbolos que representaban las alturas de las notas. 

Recién en el siglo IX, el monje benedictino Guido d'Arezzo desarrolló un 
sistema de notación musical basado en cuatro líneas y espacios, que luego evolucionó al pentagrama con cinco líneas que usamos hoy.

Solo 50 años después de la invención de la imprenta, Ottaviano Petrucci, en 1501, publica
"Harmonice Musices Odhecaton", con las primeras partituras impresas con tipos móviles, permitiendo la difusión masiva de la
música escrita.

\subsubsection{Cancioneros}

En el siglo XIX, se transmitían los acordes folclóricos de forma oral, pero para 
el siglo XX, en la música popular (tango, rock, jazz) se empieza a usar
 la notación de acordes sobre la letra de la canción.

Se difunden las revistas de acordes y letras, y luego los cancioneros impresos.
Con la llegada de internet, surgen diversas páginas y aplicaciones que revisamos en la próxima sección.

\subsubsection{Afinadores}
En el siglo XVII, Marin Mersenne formuló las leyes que gobiernan la vibración de cuerdas tensas, permitiendo
una base física para afinar cuerdas y abrió el camino hacia la acústica moderna.

$$f = \frac{1}{2L}\sqrt{\frac{T}{\mu}}$$
{\footnotesize donde $f$ es la frecuencia, $L$ es la longitud de la cuerda, $T$ es la tensión aplicada y $\mu$ es la densidad lineal de masa.}

Se afinaba con un diapasón, un metal que siempre sonaba con la misma frecuencia y
que se tomaba como nota de referencia. 

En 1939 durante la conferencia de Londres científicos y músicos acordaron que la nota La4 se 
afinaría a 440 Hz, es decir, 440 vibraciones por segundo.

En 1980 se popularizaron afinadores digitales compactos, portátiles y precisos.

\subsubsection{Metrónomos}

En 1815 Johann Maelzel patenta el metrónomo mecánico, ganándose el reconocimiento de Beethoven y popularizándose en toda Europa.

En el siglo XX llegaron metrónomos electrónicos, que permiten mayor precisión y funcionalidades adicionales como sonidos personalizables, luces intermitentes y conectividad con otros dispositivos.


\subsection{Aplicaciones Web Musicales}

Con dispositivos moviles inteligentes, escuchar un sonido y procesarlo para detectar su frecuencia es muy sencillo, igual  
que repetir un comportamiento cada un periodo constante de tiempo. Es por eso, que hubo metronomos y afinadores 
en cada celular desde el comienzo de este siglo.

Las próximas líneas transcurrirán sobre aplicaciones web que son usadas actualmente por músicos, 
pero además de señalar solamente sus aportes, se hará foco en sus debilidades.

\subsubsection{Cancioneros}

Los cancioneros online reemplazaron a los folletines ofreciendo la diversidad de la web pero copiaron un defecto.

El músico está tocando su instrumento, ajustado perfecto con el metrónomo, cuando llega al último acorde de la página 
(o de la pantalla) y lo obliga a soltar su instrumento para manipular el cancionero, perdiendo así el ritmo.

Las páginas suelen en su mayoría estar diagramadas en la pantalla pensado en maximizar el espacio para publicidad y 
no en la usabilidad del músico, por lo que ofrecen poca personalización en la vista. Esto hace que aunque muchas
ofrezcan instrucciones sobre cómo se realizan los acordes en el instrumento, o algunos implementen un rudimentario autoscroll,
 sea poco visible para el músico.

De las que tienen contenido argentino, las principales son lacuerda.net, cifraclub y acordesweb.

Y claro, muestra solo los acordes. Algunas páginas de acordes,
sin embargo, son páginas distintas!


\subsubsection{Reproductores online de video y acordes}

Estas aplicaciones web ofrecen videos con acordes sincronizados, 
como Chordify y Ultimate Guitar,
ajustan la reproducción a videos de YouTube, apenas permite editar los acordes 

\subsubsection{Editores de partituras online}

Los editores de partituras online permiten crear, editar y compartir partituras musicales a través de una interfaz web. 
musescore.com destaca por su buen balance entre contenido gratuito y de pago y su comunidad activa de usuarios,


\subsubsection{Reproductores online de audio y video}

Como YouTube y Spotify, permiten reproducir audio y video en línea, pero no están diseñados 
 para músicos que tocan juntos. Estas aplicaciones suelen permitir compartir una lista de reproducción.

\subsection{El futuro llegó, hace rato...}

Al momento de escribir esto, diciembre de 2025, los resultados del uso de la IA vienen 
avanzando enorme y exponencialmente.

Logran generar todo tipo de contenido, letras y partituras, audios y videos.

También permiten procesar audio para separar los instrumentos y transcribir partituras.
Sobresale en ese sentido la aplicación Moises.ai y Spleeter.



\subsection{Conclusiones}

Mostramos una tabla comparativa de las herramientas presentadas:

\begin{table}[h]
\centering
\scriptsize
\begin{tabular}{|l|c|c|c|c|c|c|c|c|c|}
\hline
\textbf{Característica} & 
\rotatebox{90}{\textbf{Metrónomo}} & 
\rotatebox{90}{\textbf{Diapasón}} & 
\rotatebox{90}{\textbf{Cancioneros}} & 
\rotatebox{90}{\textbf{Cancioneros Web}} & 
\rotatebox{90}{\textbf{Reprod. video y acordes}} & 
\rotatebox{90}{\textbf{Edit. partituras online}} & 
\rotatebox{90}{\textbf{Reprod. audio y video}} & 
\rotatebox{90}{\textbf{Herramientas Con IA}} & 
\rotatebox{90}{\textbf{Fogon.ar}} \\
\hline
Marca el ritmo? & Sí & No & No & No & No & No & No & Sí & Sí \\
\hline
Afina? & No & Sí & No & No & No & No & No & Sí & Sí \\
\hline
Muestra acordes? & No & No & Sí & Sí & Sí & Sí & No & Sí & Sí \\
\hline
Se actualiza con la canción? & No & No & No & Algunos & Sí & No & No & Sí & Sí \\
\hline
Muestra partitura? & No & No & No & No & No & Sí & No & Sí & Sí \\
\hline
Funciona sin Internet? & Sí & Sí & Sí & No & No & No & No & No & Sí \\
\hline
Permite estados compartidos? & No & No & No & No & No & No & Algunos & No & Sí \\
\hline
\end{tabular}
\caption{Comparativa de herramientas y aplicaciones musicales}
\label{tab:comparativa}
\end{table}

Se deduce entonces que el estado compartido es un enfoque novedoso para aplicaciones web dedicadas a músicos.

La gran cantidad de tipos de aplicaciones web musicales existentes, cada una con su solución particular, muestra la necesidad
de unificar herramientas para mejorar la experiencia del músico.

El criterio de privilegiar el aspecto educativo y de usabilidad está ausente en los cancioneros online y en la mayoría 
de las aplicaciones web musicales.