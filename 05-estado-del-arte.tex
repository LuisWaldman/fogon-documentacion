\section{Estado del Arte}
\label{sec:estado-arte}

Como es parte nuestro "negocio", describiremos algunos conceptos sobre la naturaleza
 del sonido y de la música. 
 
 Luego, comentaremos algunas tecnologías destacadas que 
 tomamos en el Fogón y sobre las novedades que llegaron con internet. También,
 repasaremos aplicaciones web con IA que ofrecen servicios relacionados con la música.

  Finalmente, explicaremos cómo nuestro enfoque es novedoso y aporta valor comparado con las herramientas antes mencionadas.

\subsection{Sobre el sonido y la música}

El sonido es una onda mecánica que se propaga a través de un medio elástico, como el aire o el agua.

Se caracteriza por propiedades físicas como la frecuencia, la amplitud y el timbre determinando 
la frecuencia el tono o la nota musical, la amplitud el volumen y otro la calidad del sonido.

Pitágoras (siglo VI a.C.) descubrió que lo que producía sonidos agradables
 entre cuerdas vibrantes era que tengan relaciones matemáticas
. Ej: (2:1)Una octava, (3:2)una quinta justa. (4:3)una cuarta justa.

Esto dio origen al estudio de la armonía: la mayoría de los humanos no podemos determinar 
la frecuencia a la que vibra una cuerda, pero si podemos distinguir si produce un sonido agradable, 
y ese sonido se produce cuando hay relaciones matemáticas simples entre las frecuencias de las notas.

Pero no solo se producen sonidos agradables con relaciones matemáticas simples entre cuerdas, 
hay un elemento fundamental en la música además de armonía y melodía: 
El ritmo es la repetición de sonidos en intervalos regulares. En la música suelen organizarse de a 
2, 3 o 4 tiempos, formando compases.


\subsection{Herramientas}

\subsubsection{Afinadores}
En el siglo XVII, Marin Mersenne formuló las leyes que gobiernan la vibración de cuerdas tensas, permitiendo
una base física para afinar cuerdas y abrió el camino hacia la acústica moderna.

$$f = \frac{1}{2L}\sqrt{\frac{T}{\mu}}$$
{\footnotesize donde $f$ es la frecuencia, $L$ es la longitud de la cuerda, $T$ es la tensión aplicada y $\mu$ es la densidad lineal de masa.}

Se afinaba con un diapasón, un metal que siempre sonaba con la misma frecuencia y
que se tomaba como nota de referencia. 

En 1939 durante la conferencia de Londres científicos y músicos acordaron que la nota La4 se 
afinaría a 440 Hz, es decir, 440 vibraciones por segundo.

En 1980 se popularizaron afinadores digitales compactos, portátiles y precisos.

\subsubsection{Metrónomos}

En 1815 Johann Maelzel patenta el metrónomo mecánico, ganándose el reconocimiento de Beethoven y popularizándose en toda Europa.

En el siglo XX llegaron metrónomos electrónicos, que permiten mayor precisión y funcionalidades adicionales como sonidos personalizables, luces intermitentes y conectividad con otros dispositivos.


\subsubsection{Pentagramas y cancioneros}

15 Siglos antes que Pitágoras, por el 2000 A.C., en la Mesopotamia se usaban tablillas de arcilla para anotar música, 
con simbolos que representaban las alturas de las notas. 
Recien en el siglo IX, el monje benedictino Guido d'Arezzo desarrollo un 
sistema de notacion musical basado en lineas y espacios, que luego evolucionó de 4 a 5 lineas.
Pocos años despues de la invencion de la imprenta, en 1501, Ottaviano Petrucci publica
"Harmonice Musices Odhecaton", con las primeras partituras impresas con tipos móviles, permitiendo la difusion masiva de la
musica escrita.
En el siglo XIX se transmitian los acordes folcloricos de forma oral, pero para 
el siglo XX, en la musica popular (tango, rock, jazz) se empieza a usar la notacion de acordes sobre la letra de la cancion.
Se difunden las revistas de acordes y letras, y luego los cancioneros impresos.
Con la llegada de internet, surgen diversas paginas y aplicaciones que revisamos en la proxima seccion.

\subsection{Aplicaciones Web Musicales}
Comparación entre otras Paginas

\subsection{El futuro llego, hace rato...}
Paginas que ofrecen servicios con IA