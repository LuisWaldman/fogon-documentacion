\subsubsection{Vista de Escenarios}
\label{sec:vista-escenarios}

La vista de escenarios (la "+1" del modelo de Kruchten) ilustra cómo las otras cuatro vistas 
trabajan juntas a través de casos de uso concretos. Documentamos los escenarios principales 
usando el formato de pruebas de aceptación de Reqnroll (Gherkin).

En este repositorio , estan desarrrolladas las pruebas automaticas 
con Playwright que validan estos escenarios:
https://github.com/LuisWaldman/fogon-pruebas/


\paragraph{Escenario 1: Tocar una canción en solitario}

\textbf{Contexto:} Un músico quiere buscar y tocar una canción desde su dispositivo.

\begin{lstlisting}[caption=Escenario: Búsqueda y reproducción de canción]
Feature: Buscar canciones
    El fogon permite buscar canciones por autor y por nombre
    pero tambien por caracteristicas tipicas como la escala, el tempo o la cantidad de acordes.

    Scenario: Busca paloma de calamaro
        Given Usuario va al fogon
        When busca "flaca calamaro"
        Then aparecen resultados relacionados con "flaca"
\end{lstlisting}


\textbf{Contexto:} Dos musicos quieren tocar una canción juntos desde sus dispositivos.

\begin{lstlisting}[caption=Escenario: Fogon]
Feature: Sesiones colaborativas
    La principal particularidad del fogon es la de mantener sesiones compartidas
    El estado compartido es la cancion, la lista de canciones y los roles de cada integrante 
    en la sesion (o fogon) 

    Scenario: Dos usuarios se conectan
        Given "Usuario1" accede a la aplicacion
        And "Usuario2" accede a la aplicacion
        And "Usuario2" inicia un fogon
        And "Usuario2" carga la cancion "Homero"
	    When "Usuario1" se une al fogon de "Usuario2"
        And "Usuario1" va a tocar
	Then "Usuario1" ve la cancion "Homero" en reproduccion
        And "Usuario2" y "Usuario1" reproducen la cancion de modo calibrado
\end{lstlisting}

