\subsubsection{Vista de Escenarios}
\label{sec:vista-escenarios}

La vista de escenarios (la "+1" del modelo de Kruchten) ilustra cómo las otras cuatro vistas trabajan juntas a través de casos de uso concretos. Documentamos los escenarios principales usando el formato de pruebas de aceptación de Reqnroll (Gherkin).

\paragraph{Escenario 1: Tocar una canción en solitario}

\textbf{Contexto:} Un músico quiere buscar y tocar una canción desde su dispositivo.

\begin{lstlisting}[caption=Escenario: Búsqueda y reproducción de canción]
Feature: Tocar canciones
  Como musico
  Quiero buscar y tocar canciones
  Para practicar con mi instrumento

Scenario: Buscar y reproducir una cancion
  Given que soy un usuario no autenticado
  When ingreso "la maquina de ser feliz" en el buscador
  Then veo una lista de resultados con esa cancion
  When selecciono "La maquina de ser feliz - Charly Garcia"
  Then veo la vista de mi instrumento (acordes para guitarra)
  When presiono el boton de reproducir
  Then comienza el autoscroll sincronizado con el metronomo
  And el compas actual se resalta automaticamente
\end{lstlisting}

\textbf{Vistas involucradas:}
\begin{itemize}
    \item \textbf{Lógica:} Gestor de Canciones, Motor de Renderizado, Motor de Reproducción
    \item \textbf{Procesos:} Proceso de renderizado UI, scheduling de eventos del metrónomo
    \item \textbf{Física:} Ejecución local en el cliente, sin comunicación con servidor
    \item \textbf{Desarrollo:} Componentes Vue (SearchView, PlayerView), Services (SongService, PlaybackService)
\end{itemize}

\paragraph{Escenario 2: Crear sesión sincronizada}

\textbf{Contexto:} Un grupo de músicos quiere ensayar tocando cada uno desde su dispositivo pero sincronizados.

\begin{lstlisting}[caption=Escenario: Sesión colaborativa sincronizada]
Feature: Sesiones sincronizadas
  Como musico lider de una banda
  Quiero crear una sesion compartida
  Para que todos toquemos sincronizados

Scenario: Crear y unirse a una sesion
  Given que soy un usuario autenticado "Luis (Guitarra)"
  When creo una nueva sesion "Ensayo Banda"
  Then obtengo un codigo de sesion "ABC123"
  
  Given que otro usuario "Ana (Voz)" ingresa el codigo "ABC123"
  When Ana se une a la sesion
  Then veo que Ana esta conectada en mi sesion
  And Ana ve el mismo estado de la sesion que yo
  
  When selecciono la cancion "Flaca - Andres Calamaro"
  Then Ana ve automaticamente la misma cancion
  And Ana ve la vista de su instrumento (letra)
  And yo veo la vista de mi instrumento (acordes)
  
  When inicio la reproduccion
  Then todos los dispositivos comienzan sincronizados
  And el compas se actualiza simultaneamente en todos
  And la diferencia de tiempo es menor a 20ms
\end{lstlisting}

\textbf{Vistas involucradas:}
\begin{itemize}
    \item \textbf{Lógica:} Gestor de Sesiones, Motor de Sincronización, Gestor de Usuarios
    \item \textbf{Procesos:} Protocolo de sincronización de relojes, WebSocket bidireccional, compensación de jitter
    \item \textbf{Física:} Clientes múltiples, servidor central coordinando, topología estrella
    \item \textbf{Desarrollo:} SessionService, SyncService, WebSocket handlers en Go
\end{itemize}

\paragraph{Escenario 3: Editar y compartir canción}

\textbf{Contexto:} Un usuario quiere corregir los acordes de una canción y compartirla con la comunidad.

\begin{lstlisting}[caption=Escenario: Edición colaborativa]
Feature: Edicion de canciones
  Como musico
  Quiero editar canciones y compartirlas
  Para contribuir a la comunidad

Scenario: Editar y publicar una cancion
  Given que estoy tocando "Como dos extranos - Los Fabulosos"
  When presiono el boton "Editar"
  Then ingreso al modo de edicion
  When cambio el acorde del compas 5 de "Am" a "Am7"
  And agrego una nueva parte "Puente" con su letra y acordes
  And modifico la secuencia para incluir el puente
  Then veo una vista previa de mis cambios
  
  When guardo la cancion
  Then la cancion se guarda localmente
  And tengo la opcion de "Publicar al servidor"
  
  When publico la cancion
  Then otros usuarios pueden encontrarla en sus busquedas
  And la cancion aparece en mi perfil como contribucion
\end{lstlisting}

\textbf{Vistas involucradas:}
\begin{itemize}
    \item \textbf{Lógica:} Gestor de Canciones (editor), Gestor de Persistencia, Gestor de Usuarios
    \item \textbf{Procesos:} Validación de formato, serialización/deserialización, API REST
    \item \textbf{Física:} Guardado local primero (IndexedDB), luego sincronización con servidor
    \item \textbf{Desarrollo:} EditorView component, SongEditor service, API endpoints en Go
\end{itemize}

\paragraph{Escenario 4: Uso offline}

\textbf{Contexto:} Un músico quiere practicar en un lugar sin internet.

\begin{lstlisting}[caption=Escenario: Operación offline]
Feature: Modo offline
  Como musico
  Quiero usar la aplicacion sin internet
  Para practicar en cualquier lugar

Scenario: Tocar canciones previamente cargadas sin conexion
  Given que previamente cargue 10 canciones con conexion
  And marque las canciones como "favoritas"
  When pierdo la conexion a internet
  And abro la aplicacion
  Then la aplicacion carga normalmente (PWA)
  And puedo acceder a mis canciones favoritas
  And puedo tocar con autoscroll y metronomo
  But no puedo crear sesiones sincronizadas
  And no puedo buscar canciones nuevas
  
  When edito una cancion en modo offline
  Then los cambios se guardan localmente
  And veo una indicacion "Pendiente de sincronizar"
  
  When recupero la conexion a internet
  Then mis cambios se sincronizan automaticamente con el servidor
\end{lstlisting}

\textbf{Vistas involucradas:}
\begin{itemize}
    \item \textbf{Lógica:} Gestor de Persistencia (prioridad local), Service Worker
    \item \textbf{Procesos:} Cache-first strategy, background sync cuando se recupera conexión
    \item \textbf{Física:} Ejecución 100\% local en el cliente
    \item \textbf{Desarrollo:} Service Worker (Workbox), IndexedDB para persistencia, sync queues
\end{itemize}

\paragraph{Cobertura de Pruebas}

Estos escenarios se implementan como pruebas de aceptación automatizadas usando:
\begin{itemize}
    \item \textbf{Framework:} Reqnroll (.NET/C\#) para BDD
    \item \textbf{Automatización:} Playwright para control del navegador
    \item \textbf{Ejecución:} CI/CD pipeline ejecuta estos tests en cada merge a main
\end{itemize}

Los escenarios validan la integración completa de las cuatro vistas arquitectónicas y garantizan que el sistema cumple con los requisitos funcionales desde la perspectiva del usuario final.
