\section{Solución Implementada}
\label{sec:solucion-implementada}

En esta seccion esta la descripcion de la solución implementada 
y una explicacion tecnica basada en el modelo de vistas 4+1 de Krutchen.

En la última vista, la vista de escenarios, 
los escribimos de igual modo que a las pruebas de aceptación en Roqnroll.

Por último, incluimos una revision de las tecnologías utilizadas.

\subsection{WWW.FOGON.AR}

Es una aplicación progresiva (PWA) y multiplataforma que permite
buscar y tocar canciones, y tambien crear, editar y compartirlas.

\begin{figure}[h]
    \centering
    \includegraphics[width=0.8\textwidth]{imagenes/tocarMaquinaSerFeliz}
    \caption{FIG 1: "La maquina de ser feliz", tocando con un video}
    \label{fig:tocar-maquina-ser-feliz}
\end{figure}

Ademas, Permite crear sesiones colaborativas para llevar en un estado compartido 
el compas y la cancion, de modo que cada musico
vea las instruccones para su instrumento, en su dispositivo, 
de manera sincronizada.

Administra la lista de reproducción y listas en general,
guardandolas localmente o en el servidor.

Ofrece tambien otras herramientas utiles para los musicos, como un afinador y 
cambios automáticos de escala.

\subsubsection{tocar}

La pagina tocar muestra la vista del instrumento para cada musico y
muestra el controlador del tiempo (Como un metronomo).


\paragraph{vistas}

\paragraph{buscar}

\paragraph{editar}

\paragraph{compartir}


\subsubsection{tocar}

\subsection{Arquitectura del Sistema}
Descripción de la arquitectura general del sistema.

\subsubsection{Vista Lógica}
qué funcionalidades ofrece el sistema
\begin{itemize}
    \item Componente 1
    \item Componente 2
    \item Componente 3
\end{itemize}


\subsubsection{Vista de procesos}
cómo se comporta dinámicamente
\begin{itemize}
    \item Componente 1
    \item Componente 2
    \item Componente 3
\end{itemize}


\subsubsection{Vista Física}
cómo se despliega en hardware (nodos, servidores, redes).
\begin{itemize}
    \item Componente 1
    \item Componente 2
    \item Componente 3
\end{itemize}

\subsubsection{Vista desarrollo}cómo se organiza el software para los desarrolladores (módulos, paquetes, capas).


\begin{itemize}
    \item Componente 1
    \item Componente 2
    \item Componente 3
\end{itemize}


\subsection{Tecnologías Utilizadas}
Listar y justificar las tecnologías seleccionadas.

\subsection{Implementación}
Describir cómo se implementó la solución propuesta.

% Ejemplo de inserción de código
\begin{lstlisting}[language=Python, caption=Ejemplo de código]
def funcion_ejemplo():
    print("Hola, mundo!")
    return True
\end{lstlisting}

% Ejemplo de figura
\begin{figure}[h]
    \centering
    % \includegraphics[width=0.8\textwidth]{imagen.png}
    \caption{Descripción de la figura}
    \label{fig:ejemplo}
\end{figure}
