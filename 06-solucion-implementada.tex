\section{Solución Implementada}
\label{sec:solucion-implementada}

En esta seccion esta la descripcion de la solución implementada 
y una explicacion tecnica basada en el modelo de vistas 4+1 de Kruchten \cite{kruchten1995}.

En la última vista, la vista de escenarios, 
los escribimos de igual modo que a las pruebas de aceptación
 en Reqnroll \cite{reqnroll2024}.

Por último, incluimos una revision de las tecnologías utilizadas.

\subsection{WWW.FOGON.AR}

Es una aplicación progresiva (PWA) y multiplataforma que permite
buscar y tocar canciones, y tambien crear, editar y compartirlas.

\begin{figure}[h]
    \centering
    \includegraphics[width=0.8\textwidth]{imagenes/tocarMaquinaSerFeliz}
    \caption{FIG 1: "La maquina de ser feliz", tocando con un video}
    \label{fig:tocar-maquina-ser-feliz}
\end{figure}

Ademas, Permite crear sesiones colaborativas para llevar
, en un estado compartido ,
el compas y la cancion, de modo que cada musico
vea las instruccones para su instrumento, en su dispositivo, 
de manera sincronizada.

Administra la lista de reproducción y listas en general,
guardandolas localmente o en el servidor.

Ofrece tambien otras herramientas utiles para los musicos, como un afinador y 
cambios automáticos de escala.

\subsubsection{tocar}

La pagina tocar muestra la vista segun el musico:

\begin{figure}[H]
\centering
\includegraphics[width=0.8\textwidth]{imagenes/TocandoMuchos}
\caption{Página tocar con distintos instrumentos en varios exploradores}
\end{figure}

En la cabecera, muestra los datos de la cancion y permite cambiar la vista y la Configuración.

A lo largo de la pantalla, la vista particular del instrumento.

Debajo, el control de reproduccion y el metrónomo.

\paragraph{icono-fogon}

Arriba a la izquierda, el icono del fogon muestra el estado general de la aplicación. Marca el pulso cuando 
estan tocando y muestra los usuarios en la sesión.

\begin{figure}[H]
\centering
\includegraphics[width=0.5\textwidth]{imagenes/iconofogones}
\caption{Iconos para los estados: desconectado, conectado, logueado, o en sesion.}
\end{figure}

\paragraph{menu}

Arriba a la derecha, el icono de menu abre las opciones para configurar la vista y el usuario.

\paragraph{vistas}


Para configurar su vista, cada musico accede desde el menu a "Ver":

\begin{figure}[H]
\centering
\includegraphics[width=0.6\textwidth]{imagenes/configuracionvista}
\caption{Configuracion de la vista}
\end{figure}

Desde aqui puede elegir entre ver letras y/o acordes o la partitura. 
Ademas, puede ajustar el tamaño de la letra, los acordes y la partitura para ajustarla a cada dispositivo.

En la opcion de reproduccion, puede elegir acompañar la reproduccion con un video o con un midi generado a travez de 
las partituras.

En las ultimas opciones, se ajuste la cantidad de columnas que se muestra en pantalla y 
si muestran instrucciones para el musico, la secuencia de acordes o la pantalla para reproducir MIDIs.


\subsubsection{Sincronizar}

Descripción de la funcionalidad de sincronización.

% Agregar contenido aquí
Cuando un usuario crea un fogón, puede invitar a otros a unirse a la sesión colaborativa.
Desde el menú, también, puede asignar distintos roles a cada usuario:

\begin{figure}[H]
\centering
\includegraphics[width=0.7\textwidth]{imagenes/captura-adminfogon}
\caption{Configuración de roles en la sesión}
\end{figure}

Todos los usuarios ven la misma canción y el mismo compás, pero cada uno ve las instrucciones
para su instrumento, en su dispositivo.
Los administradores pueden cambiar la canción y controlar la reproducción.
El director, es el único capaz de reproducir algún video o audio.

\subsubsection{Buscar y Listar}

El fogon, ademas de permitir buscar canciones por banda o por artista, tienen una serie de filtros multiopcion

\begin{figure}[H]
\centering
\includegraphics[width=0.8\textwidth]{imagenes/filtrosBusqueda}
\caption{Busqueda de "gardel" con los filtros desplegados de Escala, etiqueta y tempo}
\end{figure}

De este modo, permite buscar por:

\begin{itemize}
\item Escala
\item Etiquetas
\item Calidad
\item Tempo
\item Si tiene o no Videos o Partituras
\item Cantidad de Acordes
\item Cantidad de Partes
\item Duracion
\end{itemize}

Permite armar listas de reproduccion tanto locales como en el servidor.

\subsubsection{Editar}

Se puede editar o una cancion, o crear una nueva desde cero, eligiendo 
sus principales caracteristicas como escala, secuencia armonica y 
tipo de letra.


\begin{figure}[H]
\centering
\includegraphics[width=0.7\textwidth]{imagenes/nuevacancion}
\caption{PopUp para crear una nueva cancion}
\end{figure}

En la pantalla de edicion, se puede modificar la letra, los acordes y las partes
de la cancion. 



\begin{figure}[H]
\centering
\includegraphics[width=0.7\textwidth]{imagenes/edicionletra}
\caption{Ej de pantalla de edicion de letra}
\end{figure}

Editando la escala, es posible transponerla a cualquier otra tonalidad.

Tambien se puede agregar o editar pentagramas e importarlos desde
archivos MusicXML.

\subsubsection{Configurar}

En la pantalla de Configuración se administran los datos basicos del usuario. 

\begin{figure}[H]
\centering
\includegraphics[width=0.7\textwidth]{imagenes/configuracion}
\caption{Pantalla de configuración del usuario}
\end{figure}

Desde aqui el musico puede cambiar su nombre, el instrumento que toca y la imagen que representa su dispositivo.

Configura de que modo se crean sus sesiones: el nombre y el rol default de los nuevos usuarios que se unan a la sesion.

Permite configurar el usuario para conectarse con el servidor.

Ademas, en esta pantalla, desde la barra superior, se puede acceder a las secciones: "Hola", "Afinar", "Acerca de..." 


\begin{figure}[H]
\centering
\includegraphics[width=0.7\textwidth]{imagenes/afinador}
\caption{Pantalla de afinar instrumentos}
\end{figure}





\subsection{Arquitectura del Sistema}
Descripción de la arquitectura general del sistema.

\subsubsection{Vista Lógica}
\label{sec:vista-logica}

Presentamos el modelo de datos, la clase principal es ``Cancion''.

Luego la clase ``Aplicacion'', que funciona como orquestador de los controles de la interfaz y la conexion con el backend, 
y por ultimo la clase ``Reproductor'', que maneja la reproduccion de audio y la sincronizacion con los acordes y letras.

\paragraph{Canción}

Tiene dos propiedades de clase Letra y Acordes; además de propiedades título, artista, bpm, compás, etc.

La clase Acorde diseñada como una secuencia de partes, 
cada parte formada por una serie de acordes por compas. La propiedad ''secuencia'' es la lista
de partes como se tocan en la cancion.

[Aca figura diagrama de clases del modelo de Cancion - Ver archivo uml/diagrama-clase-cancion.puml]


\paragraph{Aplicación}

La clase Aplicacion orquesta entre el reproductor, la conexion y la interfaz de usuario. 
Maneja las clases ConexionManager para la comunicación con el servidor, Reproductor para la 
reproducción de la cancion, AutenticacionManager para el login/logout, y SesionManager para 
el manejo de sesiones colaborativas.

[Aca figura diagrama de clases del sistema de Aplicacion - Ver archivo uml/diagrama-clase-aplicacion.puml]





\subsubsection{Vista de procesos}
cómo se comporta dinámicamente
\begin{itemize}
    \item Componente 1
    \item Componente 2
    \item Componente 3
\end{itemize}


\subsubsection{Vista Física}
\label{sec:vista-fisica}

En el diagrama de despliegue vemos que nuestra aplicacion
cliente es Web y Progresiva, por lo que corre en el 
dispositivo del usuario, aunque busca en github pages
el html, css y js necesarios; ademas de los indices.

El servidor en Go esta alojado en los servidores de Render y
se conectan a una base MongoDB Atlas.

Con algunos scripts, se actualizan canciones,
por ejemplo para extraer la letra del video de
YouTube.

\begin{figure}[H]
\centering
\includegraphics[width=0.8\textwidth]{out/diaguml/diagrama-despliegue/diagrama-despliegue.png}
\caption{Diagrama de despliegue}
\end{figure}



\subsubsection{Vista de Desarrollo}
\label{sec:vista-desarrollo}

La vista de desarrollo describe la organización del software desde la perspectiva del programador, incluyendo la estructura de módulos, gestión de código fuente y estrategia de build.

\paragraph{Organización del Código}

El proyecto se organiza en dos repositorios principales:

\textbf{Repositorio Cliente (fogon-web):}
\begin{itemize}
    \item \texttt{/src/components/} - Componentes Vue.js reutilizables
    \begin{itemize}
        \item \texttt{/views/} - Vistas de instrumentos (LetraView, AcordesView, PartituraView)
        \item \texttt{/controls/} - Controles de reproducción y configuración
        \item \texttt{/tools/} - Herramientas (Afinador, Metrónomo)
    \end{itemize}
    \item \texttt{/src/services/} - Lógica de negocio e integración
    \begin{itemize}
        \item \texttt{song-service.ts} - Gestión de canciones
        \item \texttt{sync-service.ts} - Protocolo de sincronización
        \item \texttt{session-service.ts} - Gestión de sesiones
        \item \texttt{api-client.ts} - Cliente HTTP para el backend
    \end{itemize}
    \item \texttt{/src/stores/} - Estado global (Pinia/Vuex)
    \item \texttt{/src/workers/} - Web Workers para procesamiento
    \item \texttt{/src/models/} - Tipos y modelos de datos (TypeScript)
    \item \texttt{/src/utils/} - Utilidades compartidas
    \item \texttt{/public/} - Assets estáticos
\end{itemize}

\textbf{Repositorio Servidor (fogon-server):}
\begin{itemize}
    \item \texttt{/cmd/server/} - Punto de entrada de la aplicación
    \item \texttt{/internal/handlers/} - Handlers HTTP y WebSocket
    \item \texttt{/internal/services/} - Lógica de negocio
    \item \texttt{/internal/models/} - Estructuras de datos
    \item \texttt{/internal/repository/} - Capa de persistencia
    \item \texttt{/internal/sync/} - Protocolo de sincronización
    \item \texttt{/pkg/} - Paquetes reutilizables
\end{itemize}

\paragraph{Stack Tecnológico}

\textbf{Frontend:}
\begin{itemize}
    \item \textbf{Framework:} Vue.js 3 - Composition API para componentes reactivos
    \item \textbf{Lenguaje:} TypeScript - Tipado estático para mayor robustez
    \item \textbf{Build Tool:} Vite - Build rápido y HMR
    \item \textbf{Estado:} Pinia - State management
    \item \textbf{Routing:} Vue Router - Navegación SPA
    \item \textbf{Notación Musical:} VexFlow - Renderizado de partituras
    \item \textbf{PWA:} Workbox - Service Worker y cache
    \item \textbf{Estilos:} CSS3 + SCSS - Diseño responsivo
\end{itemize}

\textbf{Backend:}
\begin{itemize}
    \item \textbf{Lenguaje:} Go 1.21+ - Concurrencia nativa con goroutines
    \item \textbf{WebSocket:} gorilla/websocket - Comunicación bidireccional
    \item \textbf{HTTP Router:} chi o gorilla/mux
    \item \textbf{Base de Datos:} PostgreSQL para datos estructurados, Redis para cache/sessions
    \item \textbf{ORM:} GORM o sqlx
\end{itemize}

\textbf{Testing:}
\begin{itemize}
    \item \textbf{Unit Tests Frontend:} Vitest + Vue Test Utils
    \item \textbf{Unit Tests Backend:} Go testing package + testify
    \item \textbf{E2E Tests:} Playwright + .NET (C\#) + Reqnroll (BDD)
    \item \textbf{API Tests:} Postman/Newman
\end{itemize}

\textbf{DevOps:}
\begin{itemize}
    \item \textbf{Control de Versiones:} Git + GitHub
    \item \textbf{CI/CD:} GitHub Actions
    \item \textbf{Containerización:} Docker + Docker Compose
    \item \textbf{Deployment:} (TBD - cloud provider)
\end{itemize}

\paragraph{Gestión de Dependencias}

\begin{itemize}
    \item \textbf{Frontend:} npm/pnpm con \texttt{package.json} y lockfile
    \item \textbf{Backend:} Go modules (\texttt{go.mod}, \texttt{go.sum})
\end{itemize}

\paragraph{Estrategia de Build}

\begin{itemize}
    \item \textbf{Desarrollo:} Vite dev server con HMR para frontend, \texttt{go run} con hot-reload para backend
    \item \textbf{Testing:} Pipeline de CI ejecuta tests unitarios y E2E en cada PR
    \item \textbf{Producción:} Build optimizado con minificación, tree-shaking, y code-splitting. Docker multi-stage build para deployments eficientes.
\end{itemize}

\paragraph{Patrones de Diseño Utilizados}

\begin{itemize}
    \item \textbf{Observer:} Para reactivity en Vue.js y eventos de sincronización
    \item \textbf{Factory:} Para creación de componentes de vista según instrumento
    \item \textbf{Strategy:} Para diferentes motores de renderizado (letra, acordes, partitura)
    \item \textbf{Repository:} Para abstracción de persistencia en el backend
    \item \textbf{Singleton:} Para servicios globales (conexión WebSocket, configuración)
\end{itemize}

% TODO: Agregar diagrama de paquetes UML
% \begin{figure}[h]
%     \centering
%     \includegraphics[width=\textwidth]{imagenes/diagrama-paquetes.png}
%     \caption{Diagrama de paquetes y dependencias}
%     \label{fig:paquetes}
% \end{figure}




\subsection{Tecnologías Utilizadas}
Listar y justificar las tecnologías seleccionadas.

\subsection{Implementación}
Describir cómo se implementó la solución propuesta.

% Ejemplo de inserción de código
\begin{lstlisting}[language=Python, caption=Ejemplo de código]
def funcion_ejemplo():
    print("Hola, mundo!")
    return True
\end{lstlisting}

% Ejemplo de figura
\begin{figure}[h]
    \centering
    % \includegraphics[width=0.8\textwidth]{imagen.png}
    \caption{Descripción de la figura}
    \label{fig:ejemplo}
\end{figure}
