\section{Metodología Aplicada}
\label{sec:metodologia-aplicada}

Describiremos la metodología aplicada y la planificacion inicial,
despues comentaremos los cambios al plan inicial que 
se acordaron durante el desarrollo del proyecto.

\subsection{Metodología del trabajo}
El trabajo tendrá 6 hitos en los que se publicará la aplicación. Para
alcanzar cada uno habrá entregas incrementales periódicas, la mayoría de
una sola etapa y algunas de 2. Gestionaremos las tareas utilizando el
siguiente proyecto de GitHub: \url{https://github.com/users/LuisWaldman/projects/4}.
Detallamos a continuación las tareas que integran a cada hito.

\subsection{Plan}

\subsubsection{Tocar}
Finaliza con el sitio desplegado en \url{http://www.fogon.ar/} con
integración continua. Permite tocar algunas canciones.

\begin{tabular}{|p{0.75\textwidth}|r|}
\hline
\colorbox{gray!15}{\parbox{0.72\textwidth}{\textbf{Tareas}}} & \colorbox{gray!15}{\parbox{3.2cm}{\centering\textbf{Duración (horas)}}} \\
\hline
SetUp inicial: armar el repositorio, crear la aplicación Vue.js + TypeScript, los módulos de prueba y el despliegue automático & 12 \\
Definir JSON para canciones y armar ejemplos & 4 \\
Armar estructura de la aplicación & 4 \\
Armar menú & 2 \\
Controlador de tiempo & 8 \\
Armar pantalla tocar acordes & 4 \\
Armar pantalla tocar letra & 4 \\
Armar pantalla tocar letra y acordes & 12 \\
\hline
\textbf{Total} & \textbf{50} \\
\hline
\end{tabular}

\subsubsection{Sincronizar}
Varios usuarios pueden unirse a una sesión y reproducir juntos una lista de
canciones.

\begin{tabular}{|p{0.75\textwidth}|r|}
\hline
\colorbox{gray!15}{\parbox{0.72\textwidth}{\textbf{Tareas}}} & \colorbox{gray!15}{\parbox{3.2cm}{\centering\textbf{Duración (horas)}}} \\
\hline
Definir protocolo & 8 \\
Login de usuarios con Google (extendible) & 8 \\
SetUp inicial del servidor & 10 \\
Administración de conexiones y estructura & 30 \\
Permitir crear una sesión desde la aplicación web & 6 \\
Permitir unirse a una sesión desde la aplicación web & 6 \\
Permite ver el estado de las sesiones desde la configuración de la aplicación web & 12 \\
La canción se reproduce con el ritmo de la sesión cuando el usuario está conectado & 12 \\
Los distintos usuarios pueden mandar comandos al servidor que se reflejan en todos los conectados & 12 \\
Permite que la sesión siga viva cuando se desconecta el Admin, asignando el rol & 4 \\
\hline
\textbf{Total} & \textbf{108} \\
\hline
\end{tabular}

\subsubsection{Editar}
Agrega la administración de usuarios, que pueden editar y compartir las
canciones.

\begin{tabular}{|p{0.75\textwidth}|r|}
\hline
\colorbox{gray!15}{\parbox{0.72\textwidth}{\textbf{Tareas}}} & \colorbox{gray!15}{\parbox{3.2cm}{\centering\textbf{Duración (horas)}}} \\
\hline
Pantalla de configuración & 6 \\
Administración de usuarios & 6 \\
Configuración y despliegue de una base de datos MongoDB & 6 \\
Permitir obtener la canción de distintos repositorios: solicitud HTTP de canción pública en el sitio, IndexDB o en el servidor, canciones propias o de otros usuarios & 4 \\
Permite cambiar datos básicos de la canción & 8 \\
Permitir modificar la secuencia de acordes (el orden en que se reproducen las partes) & 4 \\
Permitir modificar los acordes de las partes: unir, cambiar, etc. & 10 \\
Permite modificar las partes: agregar, eliminar, combinar o dividir & 10 \\
Permite modificar la letra en la vista de letras y acordes & 20 \\
Permite guardar la canción en el servidor, el IndexDB o el disco duro & 10 \\
Permite cambiar la escala (cambia todos los acordes de la canción) & 8 \\
\hline
\textbf{Total} & \textbf{92} \\
\hline
\end{tabular}

\subsubsection{Buscar y listar}
El usuario puede buscar canciones en distintos repositorios y armar y
compartir listas de reproducción.

\begin{tabular}{|p{0.75\textwidth}|r|}
\hline
\colorbox{gray!15}{\parbox{0.72\textwidth}{\textbf{Tareas}}} & \colorbox{gray!15}{\parbox{3.2cm}{\centering\textbf{Duración (horas)}}} \\
\hline
Armar índices & 6 \\
Armar filtros & 6 \\
Obtener canciones mediante web scraping de internet & 12 \\
Mostrar búsqueda en un gran repositorio & 4 \\
Mostrar listas & 8 \\
Permitir guardar/editar listas & 8 \\
Definir una lista como lista de reproducción actual & 2 \\
Desde ``Tocar'' reproducir la lista de canciones actual & 6 \\
\hline
\textbf{Total} & \textbf{52} \\
\hline
\end{tabular}

\subsubsection{Tocar y editar partituras}
Muestra las partituras con VexFlow, permite editarlas y cargarlas desde
archivos XMLMusic.

\begin{tabular}{|p{0.75\textwidth}|r|}
\hline
\colorbox{gray!15}{\parbox{0.72\textwidth}{\textbf{Tareas}}} & \colorbox{gray!15}{\parbox{3.2cm}{\centering\textbf{Duración (horas)}}} \\
\hline
Agregar a JSON de la canción la posibilidad de guardar partituras para distintos instrumentos & 6 \\
Permitir desde la aplicación web definir un instrumento default para que te muestre la partitura & 3 \\
Mostrar partituras con VexFlow & 25 \\
Permitir editar partituras con VexFlow & 50 \\
Permitir agregar partituras desde archivos en el formato XMLMusic & 16 \\
Generar ejemplos de partituras & 2 \\
\hline
\textbf{Total} & \textbf{102} \\
\hline
\end{tabular}

\subsubsection{Más instrumentos y detalles}

\begin{tabular}{|p{0.75\textwidth}|r|}
\hline
\colorbox{gray!15}{\parbox{0.72\textwidth}{\textbf{Tareas}}} & \colorbox{gray!15}{\parbox{3.2cm}{\centering\textbf{Duración (horas)}}} \\
\hline
Reparar bugs de versiones anteriores & 10 \\
Vista de acordes con dibujos de cómo sería en la guitarra & 15 \\
Vista de tablaturas de riffs en guitarra & 15 \\
Vista para armónica & 15 \\
Reproducir las partituras en formato MIDI & 20 \\
\hline
\textbf{Total} & \textbf{75} \\
\hline
\end{tabular}

\subsection{Herramientas Utilizadas}

* El Visual Code como entorno de desarrollo integrado (IDE) principal para escribir y depurar el código.
* Vite como herramienta de construcción y empaquetado del proyecto.
* Git y GitHub para el control de versiones y la colaboración en el desarrollo del proyecto.
* El modelo Claude Sonnet 4.5 para desarrollo de pruebas, de codigo y de vistas.
* Jira para la gestión de proyectos y seguimiento de tareas.

\subsection{Desarrollo de la aplicación}

El desarrollo comenzó el 11 de Julio de 2025 con la 
aprobacion de la propuesta de trabajo.

En la primer entrega, se contruyó el repositorio con integracion
continua, de modo que para cada commit se ejecutaban pruebas y 
verificacion de estilo de codigo. Cada branch al main, genera
ademas el despliegue de la aplicación.


En las siguientes entregas, la de la sesion sincronizada 
y la de la edicion, vimos necesario adelantar las tareas 
relativas a pentagramas.

La integracion con VexFlow efectivamente fue compleja y
adelantar su desarrollo termino siendo una buena decicion.

En la ultima entrega planificada, agregamos listas y busquedas
tanto en el servidor como en la IndexedDB del navegador.

Durante todo el desarrollo, sentimos la necesidad de agregar
nuevos features, que desarrollamos como una nueva entrega:



\begin{tabular}{|p{0.75\textwidth}|r|}
\hline
\colorbox{gray!15}{\parbox{0.72\textwidth}{\textbf{Tareas}}} & \colorbox{gray!15}{\parbox{3.2cm}{\centering\textbf{Duración (horas)}}} \\
\hline
Sincronizar relojes por WebRTC & 20 \\
Mostrar Acordes de notas en cifrado español & 10 \\
Afinador & 10 \\
Poder sincronizar la letra de las canciones con videos de YouTube & 25 \\
Vista de acordes en teclado & 8 \\
Actualización automática de la aplicación & 2 \\
\hline
\textbf{Total} & \textbf{75} \\
\hline
\end{tabular}

El 4 de Diciembre de 2025 se realizó la entrega final, con el sitio desplegado en \url{http://www.fogon.ar/} y con todas las funcionalidades mencionadas en el plan inicial y las adicionales.