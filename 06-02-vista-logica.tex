\subsubsection{Vista Lógica}
\label{sec:vista-logica}

Presentamos el modelo de datos, la clase principal es ``Cancion''.

Luego la clase ``Aplicacion'', que funciona como orquestador de los controles de la interfaz y la conexion con el backend, 
y por ultimo la clase ``Reproductor'', que maneja la reproduccion de audio y la sincronizacion con los acordes y letras.

\paragraph{Canción}

Tiene dos propiedades de clase Letra y Acordes; además de propiedades título, artista, bpm, compás, etc.

La clase Acorde diseñada como una secuencia de partes, 
cada parte formada por una serie de acordes por compas. La propiedad ''secuencia'' es la lista
de partes como se tocan en la cancion.

[Aca figura diagrama de clases del modelo de Cancion - Ver archivo uml/diagrama-clase-cancion.puml]


\paragraph{Aplicación}

La clase Aplicacion orquesta entre el reproductor, la conexion y la interfaz de usuario. 
Maneja las clases ConexionManager para la comunicación con el servidor, Reproductor para la 
reproducción de la cancion, AutenticacionManager para el login/logout, y SesionManager para 
el manejo de sesiones colaborativas.

[Aca figura diagrama de clases del sistema de Aplicacion - Ver archivo uml/diagrama-clase-aplicacion.puml]



