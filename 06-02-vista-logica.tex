\subsubsection{Vista Lógica}
\label{sec:vista-logica}

Presentamos la clase "Cancion", que es la que se comparte y almacena. 

Luego la clase "Aplicacion" que es el punto de entrada principal del cliente.

\paragraph{Canción}

[DIAGRAMA CANCION]


\paragraph{Relaciones entre Componentes}

\begin{itemize}
    \item El \textbf{Motor de Renderizado} consulta al \textbf{Gestor de Canciones} para obtener el contenido y al \textbf{Gestor de Configuración} para las preferencias de visualización.
    
    \item El \textbf{Motor de Sincronización} notifica al \textbf{Motor de Reproducción} sobre cambios en el compás y éste actualiza el \textbf{Motor de Renderizado}.
    
    \item El \textbf{Gestor de Sesiones} coordina con el \textbf{Motor de Sincronización} para mantener el estado compartido entre dispositivos.
    
    \item El \textbf{Gestor de Persistencia} provee datos tanto al \textbf{Gestor de Canciones} como al \textbf{Gestor de Usuarios}.
\end{itemize}

% TODO: Agregar diagrama de componentes UML mostrando las relaciones entre los componentes lógicos
% \begin{figure}[h]
%     \centering
%     \includegraphics[width=\textwidth]{imagenes/diagrama-componentes-logicos.png}
%     \caption{Diagrama de componentes lógicos del sistema El Fogón}
%     \label{fig:componentes-logicos}
% \end{figure}
