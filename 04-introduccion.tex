\section{Introducción}
\label{sec:introduccion}

Tanenbaum y Van Steen definen: "Un sistema distribuido es una colección de computadoras independientes que aparece ante sus usuarios como un sistema único y coherente" (1)

La definicion coincide con la de un grupo musical que combina armonías, melodías y ritmos de modo que se escuche como un tema único y coherente.

Para hacer esto los músicos se nutren de protocolos y mecanismos de coordinación que les permiten sincronizarse y compartir información.

Los avances en la ciencia y la ingeniería fueron incorporados por los músicos: Pitágoras sintetizó la Matemática y la Armonía, la imprenta permitió la publicación de partituras, la revolución industrial el metrónomo de Maelzel.

Desde que esta internet circulan archivos con paginas de acordes que evolucionaron a paginas multimedia, archivos MIDI, aplicaciones de edición de partituras, etc.

La aplicación "El Fogón" pone a disposición de los músicos estos avances y propone un nuevo enfoque: cada músico puede acceder con su dispositivo a un Estado compartido de compás, canción, acordes, partituras, repertorio, etc.

Esto permite que varios músicos toquen juntos y en sincronía, compartiendo y actualizando información en tiempo real , pero viendo cada uno la información de su instrumento


\subsection{Motivación}
Busca hacer un aporte novedoso a la musica desde la informatica, incorporando soluciones anteriores y  
agregando un enfoque novedoso: el estado compartido entre músicos.

Ademas, busca la alta disponibilidad: Los musicos pueden acceder a la aplicacion en su dispositivo en cualquier momento, 
sin necesidad de conexion a la red.

\subsection{Objetivos}

Fogón es una solución dirigida tanto a cantantes y guitarristas aficionados como a músicos de orquestas profesionales: 
una aplicación en donde puedan buscar letras de acordes y canciones de modo intuitivo y también una herramienta que los ayude a 
ensayar y crear cosas nuevas. 


\subsubsection{Objetivo General}

Cada músico podrá ver en su dispositivo la vista de su instrumento: el cantante, la letra; el guitarrista, los acordes; 
el pianista sus partituras. Tendra Autoscroll y subrayado automático del compás actual, podra editar los tamaños de letra y acordes.

El publico podrá ver y editar una variedad de canciones publicadas en el mismo sitio. Si se loguea con su usuario,
podrá compartir sus canciones con otros usuarios. 

Varios usuarios podrán unirse en una sesión para sincronizar la lista de canciones, la canción que se está reproduciendo 
y el  compas actual: de este modo podrán organizar un ensayo, un concierto o una noche de karaoke entre amigos.


\subsubsection{Objetivos Tecnicos}
La sincronización del compás en una sesión debe ser exacta cuando un grupo de músicos está tocando: un delay de 20 ms empieza a ser perceptible por el oído humano y la latencia en internet puede ser mayor. Los distintos dispositivos se conectarán con un servidor Golang e implementarán un protocolo que combine timestamps sincronizados (basados en NTP), buffers adaptativos y compensación del jitter (variación en la latencia) para resolver esto.
El mismo servidor además por medio de HTTP intercambia los archivos de las canciones con las aplicaciones.
Todas las visitas deberán poder adaptarse a distintos dispositivos, ser configurables y extensibles: será posible incorporar vistas adicionales, como notación numérica para armónica, tablaturas para guitarra, y reproductores multimedia como YouTube o Midi.
Edicion de letra y acordes: También podrán editar las canciones mediante una interfaz intuitiva y accesible: la compleja relación entre las letras y los acordes, que además se agrupan en partes que se repiten según una secuencia podrá modificarse de una manera natural y sencilla.
Sincronicacion: Varios usuarios logueados podrán unirse en una sesión para sincronizar la lista de canciones, la canción que
 se está reproduciendo y el estado de la reproduccion.
Construir algunas herramientas nesarias para la musica como un afinador que permita afinar distintos instrumentos.
Construir herramientas para probar y "debuggear" el sistema de sincronización desarrollado.