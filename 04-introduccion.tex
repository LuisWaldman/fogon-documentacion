\section{Introducción}
\label{sec:introduccion}

Tanenbaum y Van Steen definen: ``Un sistema distribuido es una colección de computadoras independientes que aparece ante sus usuarios como un sistema único y coherente'' \cite{tanenbaum2017}

La definición coincide con la de un grupo musical que combina armonías, melodías y ritmos de modo que se escuche como un tema único y coherente.

Para hacer esto los músicos se nutren de protocolos y mecanismos de coordinación que les permiten sincronizarse y compartir información.


Los avances en la ciencia y la ingeniería fueron incorporados por los músicos: Pitágoras sintetizó la matemática y la armonía; la imprenta permitió la publicación de partituras; la revolución industrial introdujo el metrónomo de Maelzel.

Desde que existe Internet, circulan archivos con páginas de acordes que evolucionaron a páginas multimedia, archivos MIDI, aplicaciones de edición de partituras, etc.

La aplicación "El Fogón" pone a disposición de los músicos estos avances y propone un nuevo enfoque: cada músico puede acceder con su dispositivo a un Estado compartido de compás, canción, acordes, partituras, repertorio, etc.

Esto permite que varios músicos toquen juntos y en sincronía, compartiendo y actualizando información en tiempo real, pero cada uno ve la información de su instrumento.


\subsection{Motivación}
Busca hacer un aporte novedoso a la música desde la informática, incorporando soluciones anteriores y
agregando un enfoque novedoso: el estado compartido entre músicos.

Además, busca la alta disponibilidad: los músicos pueden acceder a la aplicación en su dispositivo en cualquier momento,
sin necesidad de conexión a la red.

\subsection{Objetivos}

Fogón es una solución dirigida tanto a cantantes y guitarristas aficionados como a músicos de orquestas profesionales: 
una aplicación en donde puedan buscar letras de acordes y canciones de modo intuitivo y también una herramienta que los ayude a 
ensayar y crear cosas nuevas. 


\subsubsection{Objetivo General}

Cada músico podrá ver en su dispositivo la vista de su instrumento: el cantante, la letra; el guitarrista, los acordes;
el pianista, sus partituras. Tendrá autoscroll y subrayado automático del compás actual; podrá editar los tamaños de letra y acordes.

El público podrá ver y editar una variedad de canciones publicadas en el mismo sitio. Si se loguea con su usuario,
podrá compartir sus canciones con otros usuarios.

Varios usuarios podrán unirse en una sesión para sincronizar la lista de canciones, la canción que se está reproduciendo
y el compás actual: de este modo podrán organizar un ensayo, un concierto o una noche de karaoke entre amigos.


\subsubsection{Objetivos Técnicos}

La sincronización del compás en una sesión debe ser exacta cuando un grupo de músicos está tocando: un delay de 20 ms empieza a ser perceptible por el oído humano y la latencia en Internet puede ser mayor. Los distintos dispositivos se conectarán con un servidor Golang e implementarán un protocolo que combine timestamps sincronizados (basados en NTP), buffers adaptativos y compensación del jitter (variación en la latencia) para resolver esto.

El mismo servidor, además, por medio de HTTP intercambia los archivos de las canciones con las aplicaciones.

Todas las vistas deberán poder adaptarse a distintos dispositivos, ser configurables y extensibles: será posible incorporar vistas adicionales, como notación numérica para armónica, tablaturas para guitarra y reproductores multimedia como YouTube o MIDI.

Edición de letra y acordes: también podrán editar las canciones mediante una interfaz intuitiva y accesible; la compleja relación entre las letras y los acordes, que además se agrupan en partes que se repiten según una secuencia, podrá modificarse de una manera natural y sencilla.

Sincronización: varios usuarios logueados podrán unirse en una sesión para sincronizar la lista de canciones, la canción que se está reproduciendo y el estado de la reproducción.

Construir algunas herramientas necesarias para la música, como un afinador que permita afinar distintos instrumentos.

Construir herramientas para probar y "debuggear" el sistema de sincronización desarrollado.